\documentclass{article}[12pt]
\usepackage{graphicx} % Required for inserting images
\usepackage{float}
\usepackage{amsmath}
\usepackage{hyperref}
\hypersetup{colorlinks=true, linkcolor=blue, urlcolor=blue, citecolor=blue}
\usepackage{cite}

\title{Drill on \LaTeX}
\author{Kundhavai M}
\date{27 December 2025}

\begin{document}

\maketitle

\section{Introduction}
I got the motivation from my friends that they will get attracted to the track if they see my progress. That boosted me up. I am creating this document to warm up in \LaTeX. At the end of the day I can flex to those people who don't even know 'A' in \LaTeX, "Hey! I am the Godfather (it should be Godmother, IG) of \LaTeX, so stay away from me".\\
This document is specially to get an idea of how the syntax works for each thing. Let's dive into this!

\section{Maths Saber}
\begin{equation}
    a^2 + b^2 = c^2 \label{eq:1}
\end{equation}
\begin{equation}
        \gamma^2 + \theta^2 = \omega^2 \label{eq:2}
\end{equation}
The difference between \verb|\begin{equation}| \& \verb|\begin{align}|:\\

\textbf{Equation: used for a single equation}
\begin{verbatim}
    \begin{equation}
        blah... blah... blah...
    \end{equation}
\end{verbatim}
The result:\\
\begin{equation}
    \Vec{\nabla} \cdot \Vec{E} \hspace{1em} =  \hspace{1em}\frac{\rho}{\epsilon_0} \hspace{1em}~~Gauss's \; Law \label{eq:3}
\end{equation}

\textbf{Align: used for multiple ones}
\begin{verbatim}
    \begin{align}
        blah... blah... blah...
    \end{align}
\end{verbatim}
The result:\\
\begin{align}
    \vec{\nabla} \cdot \vec{E} \quad &= \quad \frac{\rho}{\epsilon_0} && ~~Gauss's \; Law \label{eq:4}\\
    \vec{\nabla} \cdot \vec{B} \quad &= \quad 0 && ~~Gauss's \; Law \; of \; Magnetism \label{eq:5}\\
    \vec{\nabla} \times \vec{E} \quad &= \quad -\frac{\partial\vec{B}}{\partial{t}} && ~~Faraday's \; Law \; of \; Induction \label{eq:6}\\
    \vec{\nabla} \times \vec{B} \quad &= \quad \mu_0\left(\epsilon_0\frac{\partial\vec{E}}{\partial{t}} + \vec{J}\right) && ~~Ampere's \; Circuital \; Law \label{eq:7}
\end{align}
Equations \ref{eq:4}, \ref{eq:5}, \ref{eq:6}, \ref{eq:7} look toxic and they are famous in physics books.

\subsection{Matrix horror}
\begin{matrix}
    a_{11}&a_{12}&\dots&a_{1n}\\
    a_{21}&a{22}&\dots&a_{2n}\\
    \vdots&\vdots&\ddots&\vdots\\
    a_{m1}&a_{m2}&\dots&a_{mn}
\end{matrix}
\begin{equation}
    \begin{pmatrix}
        a_{11}&a_{12}&\dots&a_{1n}\\
        a_{21}&a{22}&\dots&a_{2n}\\
        \vdots&\vdots&\ddots&\vdots\\
        a_{m1}&a_{m2}&\dots&a_{mn}
    \end{pmatrix}
    \begin{bmatrix}
        v_1\\
        v_2\\
        \vdots\\
        v_n
    \end{bmatrix}   
    =
    \begin{matrix}
    w_1\\
    w_2\\
    \vdots\\
    w_n
    \end{matrix}
\end{equation}

\section{Tables and Figures}
\begin{verbatim}
\begin{table}[hbt!]
    \centering
    \caption{This is a table that shows how to create different lines as well as different justifications}
    \begin{tabular}{|l||c|c|r|}
        \hline
        $x$&1&2&3\\
        \hline
        $f(x)$&4&8&12\\
        f(x)&4&8&12\\
        \hline
    \end{tabular}
\end{table}
\end{verbatim}
\begin{table}[hbt!]
    \centering
    \caption{This is a table that shows how to create different lines as well as different justifications}
    \begin{tabular}{|l||c|c|r|}
    \hline
    $x$&1&2&3\\
    \hline
    $f(x)$&4&8&12\\
    f(x)&4&8&12\\
    \hline
    \end{tabular}
\end{table}
\begin{table}[hbt!]
    \centering
    \caption{This is a table that shows the meaning of [hbt!]}
    \begin{tabular}{|l|l|p{10cm}|}
    \hline
    Specifier&Meaning&Description\\
    \hline
    $h$&Here&Place the table at the exact location where it appears in the source code, if there is enough space.\\
    $b$&Bottom&Place the table at the bottom of the current or next available page.\\
    $t$&Top&Place the table at the top of the current or next available page.\\
    $!$&Force&	Instructs LaTeX to ignore certain internal restrictions (such as the maximum number of floats allowed on a single page) to try harder to fit the table in the specified positions.\\
    \hline
    \end{tabular}
\end{table}
\begin{figure}[H]
    \centering
    \includegraphics[width=0.5\linewidth]{trollsmile.png}
    \caption{Troll Smile}
    \label{fig:trollsmile}
\end{figure}

\section{Bibliography}

Well, I'm bored to the core. Let me wind this up. I wasted three hours thinking and writing things, it used almost all my brain cells. I'm done for today.\\
\begin{thebibliography}{99}
\bibitem{The Doc}
\href{https://guides.nyu.edu/LaTeX/exercises}{Exercises for \LaTeX}
\end{thebibliography}
\end{document}
